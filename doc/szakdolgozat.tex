\documentclass[twoside, a4paper, 12pt]{article}
\usepackage{thesis-style}

\usepackage{hyperref}
\usepackage{xcolor}

\usepackage{graphics}
\graphicspath{ {uml/png/} }

\usepackage{amsmath}
%\usepackage{mathbbol}
\usepackage{amsfonts}
\usepackage{amssymb}

% Töltsd ki a saját szakdolgozatod adataival
\def\CIM{Audiovizualiz\'aci\'o val\'os id\H oben\newline\newline Vulkan API-val}
\def\SZERZO{Vida P\'eter}
\def\VEDESEVE{2018}

\def\TANSZEK{Programozási Nyelvek és Fordítóprogramok Tanszék}
\def\TEMAVEZETO{Gera Zolt\'an}
\def\TEMAVEZETOBEOSZTAS{Tan\'arseg\'ed}


\title{\CIM}
\author{\SZERZO}
\date{\VEDESEVE}

\definecolor{light-gray}{gray}{0.95}
\newcommand{\code}[1]{\colorbox{light-gray}{\texttt{#1}}}

\newcommand{\graf}{\mathrm{graf}}
\newcommand{\conj}[1]{\overline{#1}}
%inline fraction
\newcommand{\infrac}[2]{#1/#2}
\newcommand{\vecc}[1]{\mathbf{#1}}
\newcommand{\hz}{\mathrm{Hz}}
\newcommand{\scalmul}[2]{\left<#1,#2\right>}

\setcounter{tocdepth}{5}
\setcounter{secnumdepth}{6}

\begin{document}
\pagestyle{empty}

% belső fedőlap
\include{tex/fedolap}
\cleardoublepage

% a belső fedőlap utáni lap a témabejelentő

% tartalomjegyzék
\tableofcontents
\cleardoublepage

\pagestyle{plain}
\setcounter{page}{1}

% tartalom
% Ajánlott minden fő fejezetet külön fájlba írni, pl.:


\section{Bevezet\H o}
\paragraph{} 
Inform\'aci\'o minden\"utt jelen van k\"or\"ul\"ott\"unk. A technol\'ogia fejl\H od\'es\'evel egyre t\"obbet \'es t\"obbet tudtunk bel\H ole digitaliz\'alni. Ez\'altal az adat rendelkez\'esre \'all, a k\'erd\'es az marad, hogy ezt hogyan tudjuk felhaszn\'alni.

Az ut\'obbi id\H oben m\'egink\'abb el\H ot\'erbe ker\"ult a vide\'ok\'arty\'ak nem csak j\'at\'ekokban val\'o kihaszn\'al\'asa, hanem nagym\'ert\'ekben t\"obbsz\'alas\'ithat\'o probl\'em\'ak nagy bemenetre val\'o felhaszn\'al\'asa. 
Ugyanakkor a nagyobb sz\'am\'it\'asi kapacit\'as nem sokat \'er, ha a programoz\'o nem tudja kihaszn\'alni. Az OpenGL egyre ink\'abb alulmaradt a DirectX-szel szemben, mind haszn\'alhat\'os\'ag, mint teljes\'itm\'eny szempontj\'ab\'ol.

Az OpenGL-\'ert felel\H os \href{https://www.khronos.org/}{Khronos Group} a 2015-\"os GDC\footnote{\href{http://www.gdconf.com/}{Game Developers Conference}}-n bejelentette a Vulkan-t, mint egy alacsonyszint\H u, cross-platform sz\'am\'it\'asi \'es 3D grafika API.

\paragraph{Vulkan}
Maga az API a filoz\'ofi\'aj\'aban arra t\"orekszik, hogy min\'el t\"obb feladatot a fejleszt\H o kez\'ebe adjon. Ennek t\"obb el\H onye is van; egyr\'eszt kisebb feladat h\'arul a gy\'art\'okra, mert egyszer\H ubb lesz a driver, mivel kevesebb feladatot kell ell\'atnia, m\'asr\'eszt a fejleszt\H o az, aki tiszt\'aban van az alkalmaz\'as fel\'ep\'it\'es\'evel \'es m\H uk\"od\'es\'evel, \'igy min\'el k\"ozelebb ker\"ul a hardverhez, ann\'al t\"obb lehet\H os\'ege van az optimaliz\'aci\'ora. \newline
Cser\'ebe mivel t\"obb feladatk\"or h\'arul az alkalmaz\'as k\'esz\'it\H o(i)re, maga a fejleszt\'esi folyamat lesz hosszabb.

%\paragraph{Motiv\'aci\'o}


\section{Felhaszn\'al\'oi dokument\'aci\'o}
\subsection{Feladat}
A program azt feladatot hivatott ell\'atni, hogy mikrofonbemenetr\"ol \'erkez\H o hangjelet feldolgozzon, majd ezt az eredm\'enyt vizu\'alisan megjelen\'itse. Mindezt \'ugy, hogy az id\H o ami eltelik a jel \'eszlel\'ese \'es az eredm\'eny megjelen\'ese k\"oz\"ott minim\'alis legyen.

A megjelen\'it\'es egy h\'aromdimenzi\'os spektogramon t\"ort\'enik. Ezen az $x$-tengelyen az id\H o, az $y$-tengelyen a frekvencia, a $z$-tengelyen pedig a hangban tal\'alt adott frekvenci\'aj\'u hull\'am amplit\'ud\'oja jelenik meg.

\subsection{Hangfeldolgoz\'as}
A jel feldolgoz\'asa vide\'ok\'arty\'an v\'egzett diszkr\'et Fourier-transzform\'aci\'oval t\"ort\'enik. A transzform\'aci\'o a hangjelet az id\H o f\"uggv\'eny\'er\H ol a frekvencia f\"uggv\'eny\'eve alak\'itja. 

Tekintettel arra, hogy a diszkr\'et Fourier-transzform\'aci\'o kifejezetten m\H uveletig\'enyes algoritmus ($\ordo{n^2}$) a gyakorlatban \'altal\'aban gyors Fourier-transzform\'ac\'iot haszn\'alnak az ilyen feladatok megold\'as\'ara, amellyel $\ordo{n\cdot\log n}$-re lehet ezt jav\'itani.

Azonban a processzorral ellent\'etben, ahol kevesebb mag, de azok magasabb \'orajelen, a GPU-ban rengeteg sz\'am\'it\'asi egys\'eg van, amelyek egym\'ast\'ol f\"uggetlen\"ul p\'arhuzamosan tudnak m\H uk\"odni, viszont alacsonyabb \'orajelen. \'Igy ha a kimenet egyes  komponenseit k\"ul\"on sz\'am\'it\'asi egys\'eg sz\'amolja p\'arhuzamosan, akkor elm\'eletileg $\ordo{n}$-es m\H uveletig\'enyr\H ol besz\'elhet\"unk, ami m\'egink\'abb hozz\'aj\'arul a kis k\'es\'es\H u megjelen\'it\'eshez. 


\subsection{Technol\'ogi\'ak}
\subsubsection{Megjelen\'it\'es}
Az alkalmaz\'as egy darab megjelen\'it\H o ablakb\'ol \'all. Az ablakkezel\'es\'ert a ny\'ilt forr\'ask\'od\'u \href{http://www.glfw.org/}{GLFW} multi-platform k\"onyvt\'ar felel\H os.

TODO: k\'ep a fut\'asr\'ol

\subsubsection{Hangkezel\'es}
A mikrofon bemenet kezel\'es\'et a PortAudio cross-platform k\"onyvt\'ar v\'egzi. Ezt haszn\'alja t\"obbek k\"oz\"ott az \href{http://audacity.sourceforge.net}{Audacity} hangfelvev\H o \'es -v\'ag\'o program vagy a sokak \'altal kedvelt \href{http://www.videolan.org/vlc/}{VLC Media Player} is. A teljes lista megtal\'alhat\'o a \href{http://www.portaudio.com/apps.html}{weboldalukon}.

Ugyanakkor az implement\'aci\'o (\ref{audiohandling}) lehet\H ov\'e teszi, hogy ennek a k\"onyvt\'arnak a lecser\'el\'ese csak a program kis r\'esz\'et \'erintse.
A j\"ov\H oben \'erdemes lehet megvizsg\'alni, hogy k\"ulonb\"oz\H o hangkezel\'esi megold\'asok hogyan befoly\'asolj\'ak a program teljes\'itm\'enykritikus r\'eszeit.

\subsubsection{Vide\'ok\'artya kezel\'es}
A vide\'ok\'arty\'aval val\'o munk\'at a bevezet\H oben is eml\'itett Vulkan cross-platform sz\'am\'it\'asi \'es grafikai API teszi lehet\H ov\'e. A program keret\'eben az API mindk\'et funkcionalit\'asa felhaszn\'al\'asra ker\"ult.



\subsection{Rendszerk\"ovetelm\'enyek}\label{runrequirements}
A fejleszt\'es Linuxon, 64 bites AntergOS disztrib\'uci\'on zajlott. 
\newline
K\'et konfigur\'aci\'on lett tesztelve:
\begin{itemize}
	\item Egy asztali g\'epen
		\begin{itemize}
			\item Intel Core-i5 4460
			\item NVIDIA GeForce GTX 960
		\end{itemize}
	\item Egy laptopon
		\begin{itemize}
			\item Intel Core-i5 4300U
			\item Intel HD Graphics 4400
		\end{itemize}
\end{itemize}

Mivel a felhaszn\'alt eszk\"oz\"ok mindegyike cross-platform, \'igy elm\'eletben Windowson is ford\'ithat\'o \'es futtathat\'o a program, de ez nem ker\"ult tesztel\'esre.

\subsubsection{Vide\'ok\'artya driver}\label{gpudriver}
A Vulkan API t\'amogatotts\'ag\'arol sz\'ol\'o t\'abl\'azat \href{https://en.wikipedia.org/wiki/Vulkan_(API)\#Compatibility}{itt} tal\'alhat\'o.

A haszn\'alt Linux disztrib\'uci\'o csomagjai k\"oz\"ott \'erdemes a \code{vulkan} kulcssz\'ora r\'akeresni \'es a haszn\'alt hardvernek megfelel\"o csomagot/csomagokat feltelep\'iteni vagy a gy\'art\'o oldal\'ar\H ol a legfrissebb drivert feltelep\'iteni.

Tov\'abba Vulkan alkalmaz\'asok futtat\'as\'ahoz kell a Vulkan r\'eteg-\'es driver kezel\H o konyvt\'ar. (Debian alap\'u rendszereken \code{libvulkan1}, Arch alap\'uakon \code{vulkan-icd-loader}).

\subsection{Futtat\'as}
\begin{figure}[h]
	\includegraphics[width=\textwidth]{img/user/screenshot}
	\centering
	\caption{K\'eperny\H ok\'ep a programr\'ol fut\'as k\"ozben.}
\end{figure}

Ha a futtat\'ashoz sz\"uks\'eges felt\'etelek adottak (\ref{runrequirements}) a program a \code{bin/release} mapp\'aban tal\'alhat\'o \code{.out} kiterjeszt\'es\H u f\'ajl elind\'it\'as\'aval futtathat\'o. 



\section{Fejleszt\H oi dokument\'aci\'o}


\subsection{A feladat r\'eszletes le\'ir\'asa}
\subsubsection{Modell}
Adott egy hangull\'am, ami a leveg\H o nyom\'asa az id\H o f\"uggv\'eny\'eben:
\(s: \mathbb{T} \rightarrow \mathbb{P} \) \newline
A modell\"unkben $\mathbb{T}:=[0,+\infty), \mathbb{P}:=[-1,1] $ \newline
A hanghull\'am, mint \"osszetett rezg\'es fel\'irhat\'o k\"ul\"onb\"oz\H o amplit\'ud\'oj\'u, f\'azis\'u \'es frekvenci\'aj\'u harmonikus rezg\'esek (r\'eszhangok) \"osszegek\'ent, azaz: \newline
\( s(t) = \sum_i r_i\cdot\sin{(f_i\cdot t + \varphi_i)} \), ahol $r_i$ az amplit\'ud\'oja, $f_i$ a frekvenci\'aja, $\varphi_i$ pedig a f\'azisa az $i$-edik r\'eszhangnak.
\subsubsection{Specifik\'aci\'o}
\paragraph{Bemenet}
A modellben szerepl\H o $s$ f\"uggv\'enyb\H ol $\Delta t$ id\H o alatt egy $N$ elem\H u mint\'at v\'etelez\"unk: $s_1,\dots,s_{N}$, \'es \newline
\( \begin{array}{rcl}
s_i := s(x_i) & \text{, ahol } & i\in[1..N],\\
&&x_i\in\{ t_0 = x_0 < x_1 < \dots < x_{N-1} = t_0 + \Delta t \} \text{, tov\'abb\'a } \\
&&x_j - x_{j-1} = \frac{\Delta t}{N}\ (j\in[2..N] ) 
\end{array} \)
\paragraph{Kimenet}
Legyen $\mathbb{F}:=\left[ 0..\frac{N}{2} \right]$\newline>
Ekkor a kimenet egy olyan $R_{s,t_0}:\mathbb{F} \rightarrow [0,+\infty)$ f\"uggv\'eny, amely a frekvencia f\"uggv\'eny\'eben megadja az $s\rvert_{[t_0,t_0+\Delta t]}$ hanghull\'amban az adott frekvenci\'aj\'u r\'eszhang amplit\'ud\'oj\'at, azaz $R_{s,t_0}(f_i) = r_i$.
\paragraph{\'Abr\'azol\'as}
Maga a spektrogram az al\'abbi f\"uggv\'ennyel \'irhat\'o le:
\[ \begin{aligned}[rl]
S&:\mathbb{T}\times\mathbb{F} \rightarrow [0,+\infty), \\
S&(t,f):=R_{s,t}(f)
\end{aligned}
\]
Ekkor az \'abr\'azol\'as az $S$ f\"uggv\'eny grafikonj\'anak, azaz
\[
\graf S = \left\{ \big( (t,f) ,S(t,f) \big) : t\in [0,+\infty), f\in\mathbb{F} \right\}
\]




\subsection{Implement\'aci\'o}

\subsubsection{Architekt\'ura}
Az alkalmaz\'as k\'etr\'eteg\H u modell-n\'ezet architekt\'ura szerint k\'esz\"ult.
Ezek m\H uk\"od\'es\'et a \code{VisaulizationApplication} oszt\'aly k\"oti \"ossze \'es ir\'any\'itja.

\subsubsection{Alkalmaz\'as}
\begin{figure}[h]
	\includegraphics[width=\textwidth]{VisualizationApplication__VisualizationApp_0}
	\centering
	\caption{Az alkalmaz\'as oszt\'alydiagramja \'es a csomagkapcsolatok}
\end{figure}

Az alkalmaz\'as nev\'et a konstruktorban lehet be\'all\'itani, majd a \code{run} met\'odussal elind\'itani. \'Igy a \code{src/main.cpp} f\'ajlban tal\'alhat\'o \code{main} f\"uggv\'enynek csak az alkalmaz\'as futtat\'asa \'es a program sor\'an eldobott kiv\'etelek legv\'egs\H o elkap\'asa lesz a feladata.
Maga az alkalmaz\'as a k\"ovetkez\H o v\'altoz\'okon kereszt\"ul param\'eterezhet\H o:
\begin{itemize}
	\item \code{WIDTH} \'es \code{HEIGHT}: a megjelen\'it\H o ablak kezdeti m\'erete.
	\item \code{windowSize}: mekkora legyen a minta, amit elemz\"unk; a specifik\'aci\'oban $N$
\end{itemize}

\subsubsection{N\'ezet}
\begin{figure}[h]
	\includegraphics[width=\textwidth]{View__ViewClassDiagram_7}
	\centering
	\caption{A n\'ezet r\'eteg}
\end{figure}
Az alkalmaz\'as fel\"ulete egy darab megjelen\'it\H o ablakb\'ol \'all, ez a prezent\'aci\'os fel\"ulet.
Ezt a \code{WindowHandler} oszt\'aly biztos\'itja, a GLFW k\"onyvt\'ar seg\'its\'eg\'evel. \newline
A konstruktor\'aban inicializ\'alja a GLFW-t, l\'etrehozza \'es be\'all\'itja az ablakot, be\'all\'itja a hibakezel\H o f\"uggv\'eny\'et. \newline
A \code{getGLFWextensions} f\"uggv\'ennyel inform\'aci\'ot szerezhet\"unk a k\"onyvt\'art\'ol, hogy milyen kieg\'esz\'it\H o funkcionalit\'asokra lesz sz\"uks\'eg\"unk a z\"okken\H omentes m\H uk\"od\'eshez. Ezt felhaszn\'aljuk amikor egy kompatibilis fizikai eszk\"ozt keres\"unk.

\subsubsection{Modell}
\begin{figure}[h]
	\includegraphics[width=\textwidth]{Model__ModelClassDiagram_1}
	\centering
	\caption{A modell r\'eteg kapcsolati \'abr\'aja}
\end{figure}
A modell r\'eteg tov\'abbi r\'eszekre bonthat\'o.

\paragraph{Hangkezel\'es}
\begin{figure}[h]
	\includegraphics[width=\textwidth]{Model__AudioHandler_5}
	\centering
	\caption{Az \code{AudioHander} oszt\'alydiagramja}
\end{figure}
A hang kezel\'es\'et az \code{AudioHandler} oszt\'aly biztos\'itja.
A konstruktor\'aban lefoglal a t\'arol\'oknak helyet \'es inicializ\'alja a \code{PortAudio} k\"onyvt\'art.
H\'arom param\'eterrel tudjuk a m\H uk\"od\'es\'et finomhangolni:
\begin{itemize}
	\item A buffer m\'erete
	\item A mintav\'etelez\'esi r\'ata
	\item H\'any csatorn\'an t\"ort\'enik a mintav\'etelez\'es
\end{itemize}
Az ut\'obbi kett\"onek csak mikrofonbemenet haszn\'alata eset\'en van jelent\H os\'ege.
Az \code{overlap} adattag azt szab\'alyozza, hogy a v\'etlezett mint\'ak mekkora fed\'esben legyenek. $\code{overlap}\in[0..\code{bufferSize})$
TODO: a r\'at\'at belevenni a f\'ajlbeolvas\'asba
\newline
Ezek ut\'an h\'aromf\'elek\'epp tud hangadatot szolg\'altatni:
\begin{enumerate}
	\item Harmonikus rezg\'esek line\'aris kombin\'aci\'ojak\'ent el\"o\'all\'it egy mesters\'eges \"osszetett hangot
		\begin{enumerate}
			\item A \code{generateTestAudio} f\"uggv\'enynek megadjuk, hogy mekkora intervallumon milyen frekvenci\'aj\'u \'es amplit\'ud\'oj\'u harmonikus rezg\'eseket \'all\'itson el\H o
			\item Majd a \code{getNormalizedTestAudio} f\"uggv\'ennyel tudjuk az el\H o\'all\'itott hang egy prefix\'et elk\'erni. A f\"uggv\'eny t\"orli ezt a prefixet.
		\end{enumerate}
	\item M\'ar l\'etez\H o \code{wav} f\'ajlt bet\"olt hangbemenetk\'ent
		\begin{enumerate}
			\item A \code{loadTestWAV} f\"uggv\'ennyel beolvassa a f\'ajlt.
			\item A \code{getNormalizedTestAudio} f\"uggv\'ennyel a beolvasott f\'ajlt egy prefix\'et elk\'erni. A f\"uggv\'eny t\"orli ezt a prefixet.
		\end{enumerate}
	\item Mikrofon \'altal \'eszlelt hangot haszn\'aljon.
		\begin{enumerate}
			\item A \code{startRecording} met\'odussal elind\'itja a PortAudio streamet.
			\item Ezut\'an a \code{getMicrophoneAudio} f\"uggv\'ennyel tudunk az \'eszlelt adatb\'ol olvasni.
		\end{enumerate}
\end{enumerate} 

TODO: interf\'essz\'e alak\'it\'as, \'es k\"ul\"on TestAudioHandler illetve MicAudioHandler.

\paragraph{Hangfeldolgoz\'as}
\begin{figure}[h]
	\includegraphics[width=\textwidth]{Model__VulkanCompute_3}
	\centering
	\caption{A \code{VulkanCompute} oszt\'alydiagramja}
\end{figure}
A hangfeldolgoz\'as a specifik\'aci\'oban eml\'itett m\'odon diszkr\'et Fourier-transzform\'aci\'oval t\"ort\'enik, ami a \code{VulkanCompute} oszt\'aly seg\'its\'eg\'evel vide\'ok\'arty\'an t\"ort\'enik.
A konstruktor\'aban l\'etrehozza a m\H uk\"od\'es\'ehez sz\"uks\'eges eszk\"oz\"oket. 
\begin{itemize}
	\item A logikai eszk\"ozt, amin kereszt\"ul a program a fizikai eszk\"ozzel kommunik\'alni tud.
	\item A logikai eszk\"ozh\"oz tartoz\'o sort, ahova az utas\'it\'asokat lehet k\"uldeni.
	\item Mem\'ori\'at allok\'al az eszk\"ozon.
	\item Buffereket hoz l\'etre a mem\'oria el\'er\'es\'ere.
	\item L\'etrehozza az er\H oforr\'as le\'ir\'okat.
	\item Bet\"olti a sz\'am\'it\'ast defini\'al\'o shadert.
	\item Sz\'am\'it\'asi szerel\H oszalagot hoz l\'etre.
	\item Command pool-t kre\'al \'es command buffer-t foglal bel\H ole.
	\item Felveszi a majd v\'egrehajtand\'o parancsokat a command buffer-be.
\end{itemize}

Ezek ut\'an haszn\'alatra k\'esz, amely a k\"ovetkez\H o sorrendben t\"ort\'enik:
\begin{enumerate}
	\item Adatok felt\"olt\'ese a mem\'ori\'aba a \code{copyDataToGPU} f\"uggv\'eny seg\'its\'eg\'evel.
	\item A sz\'am\'it\'as futtat\'asa a \code{runCommandBuffer} met\'odus megh\'iv\'asa \'altal.
	\item A sz\'amolt eredm\'eny kiolvas\'as\'a a \code{readDataFromGPU} f\"uggv\'ennyel.
\end{enumerate}
TODO: A sz\'amol\'o \'es a rajzol\'o haszn\'aljon k\"oz\"os mem\'ori\'at.

\paragraph{Rajzol\'as}
\begin{figure}[h]
	\includegraphics[width=\textwidth]{Model__VulkanGraphics_4}
	\centering
	\caption{A \code{VulkanGraphics} oszt\'alydiagramja}
\end{figure}
A kisz\'amolt amplit\'ud\'o \'ert\'ekeket a \code{VulkanGraphics} oszt\'aly seg\'its\'egevel tudjuk az ablakon megjelen\H o rajzz\'a alak\'itani. 
Inicializ\'al\'askor a sz\'am\'it\'asi egys\'eghez hasonl\'oan hoz l\'etre a m\H uk\"od\'es\'ehez sz\"uks\'eges eszk\"oz\"oket, csak mivel a megjelen\'it\'es \"osszetettebb feladat, ezek kieg\'esz\"ulnek, illetve m\'odusulnak.
\begin{itemize}
	\item L\'etrehoz egy rajzol\'as\'ert felel\H os logikai eszk\"ozt, amin kereszt\"ul a fizikai eszk\"ozzel lehet kommunik\'alni.
	\item Be\'all\'itja a rajzol\'o \'es a prezent\'al\'o sorokat.
	\item L\'etrehoz egy \code{swapchain}-t. Ez felel\H os a bufferek kezel\'es\'e\'ert. A jelenlegi implement\'aci\'o, amennyiben a hardver is t\'amogatja \code{double buffering}-et haszn\'al. Rajzol\'askor a \code{swapchain}-t\H ol kell k\'epet (pontosabban a k\'ep indexet) k\'erni, amire majd a rajz ker\"ul.
	\item L\'etrehozza a t\'enyleges k\'epeket, amikre rajzolni lehet.
	\item Be\'all\'it egy renderel\H o menetet.
	\item Defini\'alja az er\H oforr\'as le\'ir\'o s\'em\'at.
	\item Defini\'alja a szerel\H oszalag s\'em\'aj\'at.
	\item L\'etrehozza a grafikus szerel\H oszalagot.
	\item L\'etrehozza a \code{frame buffer}-eket, ami \"osszek\"oti a k\'epekkel a renderel\'est.
	\item L\'etrehozza az utas\'it\'as pool-t.
	\item Mem\'ori\'at allok\'al az eszk\"oz\"on a vertexeknek, indexeknek \'es a transzform\'aci\'os m\'atrixoknak, majd \"osszekapcsolja \H oket a megfelel\H o bufferekkel.
	\item Inicializ\'alja az er\H oforr\'asle\'ir\'o pool-t, majd allok\'alja bel\H ole a t\'enyleges er\H oforr\'as le\'ir\'okat.
	\item Lefoglal az utas\'it\'as pool-b\'ol utas\'it\'as buffert.
	\item L\'etrehozza a rajzol\'as folyamat\'an a helyes fut\'asi sorrend biztos\'it\'as\'ara haszn\'alt szemaforokat. 
\end{itemize}
Miut\'an sikeresen l\'etrej\"ott a \code{VulkanGraphics} objektum, a k\"ovetkez\H o m\'odon t\"ort\'enhet a haszn\'alata.
\begin{enumerate}
	\item Az \code{appendVertices} met\'odussal felt\"olti a vertexeket a \code{vertices} vektorba, illetve kisz\'amolja a rajzoland\'o h\'aromsz\"ogek cs\'ucspontvertexeinek indexeit \'es azokat felt\"olti az \code{indices} vektorba.
	\item Az \code{updateUniformBuffer} f\"uggv\'eny friss\'iti a transzform\'aci\'os m\'atrixokat.
	\item A \code{drawFrame} elj\'ar\'as pedig kirajzolja az aktu\'alis k\'epkock\'at.
		\begin{enumerate}
			\item Elk\'eri a k\'ep index\'et a swapchain-t\H ol. (aszinkron)
			\item Felt\"olti a \code{vertices} \'es az \code{indices} t\"omb tartalm\'at a a vide\'ok\'artya mem\'ori\'aj\'aba.
			\item Felveszi a utas\'it\'asbuffer-be a v\'egrehajtand\'o utas\'it\'asokat.
			\item Elk\"uldi a grafikus sornak az utas\'it\'ast v\'egrehajt\'asra. (aszinkron)
			\item Felk\'esz\"ul az elk\'esz\"ult k\'ep prezent\'al\'as\'ara, \'es amint a \code{renderFinishedSemaphore} jelez, hogy elk\'esz\"ult a renderel\'es, kirakja megjelen\'it\H o fel\"uletre az elk\'esz\"ult k\'epet.
		\end{enumerate}
\end{enumerate}

\paragraph{VulkanFrame}
\begin{figure}[h]
	\includegraphics[width=\textwidth]{Model__VulkanFrame_2}
	\centering
	\caption{A \code{VulkanFrame} oszt\'alydiagramja}
\end{figure}
Ez az oszt\'aly felel\H ose, hogy a sz\'amol\'o \'es a rajzol\'o r\'esz k\"oz\"osen haszn\'alt eszk\"ozeit \"osszefogja. Hozza l\'etre a Vulkan Instance-t, illetve v\'alasztja ki az alkalmaz\'as ig\'enyeinek megfelel\H o fizikai eszk\"ozt. Illetve ez szerzi meg az ablakkezel\H ot\H ol a rajzol\'asi fel\"uletet is, amit majd a rajzol\'o oszt\'aly a prezent\'al\'ashoz haszn\'al.

\paragraph{A utility n\'evt\'er}
\begin{figure}[h]
	\includegraphics[width=\textwidth]{Model__Utility__Utility_6}
	\centering
	\caption{A \code{utility} n\'evt\'er tartalma}
\end{figure}
Ebben a csomagban seg\'edf\"uggv\'enyek tal\'alhat\'oak a Vulkan-t haszn\'al\'o oszt\'alyokhoz. Azok a met\'odusok ker\"ultek ide, amelyek a sz\'am\'it\'asi \'es a grafikai r\'eszben is m\H uk\"od\'es\"ukben hasonl\'oak, \'igy megfelel\H o param\'eterez\'essel \'altal\'anos\'itani lehetett \H oket vagy csak logikailag valamilyen szinten elk\"ul\"on\'ithet\H oek voltak a f\H o oszt\'alyt\'ol.

H\'arom tov\'abbi r\'eszre bonthat\'o.
\subparagraph{pipeline}
A szerel\H oszalagok l\'etrehoz\'as\'aval kapcsolatos seg\'edf\"uggv\'enyek.

\subparagraph{memory}
Vulkan mem\'oriakezel\'es\'e\'ert felel\H os met\'odusok.

\subparagraph{swapchain}
A swapchain l\'etrehoz\'as\'at seg\'iteni hivatott elj\'ar\'asok \'es f\"uggv\'enyek.

\subsubsection{Fejleszt\'esi lehet\H os\'egek}
Hol is kezdjem...

%these two may be the same
%\input{tex/dev/application}


\subsection{Fejleszt\'es}
\subsubsection{El\H ok\"ovetelm\'enyek}
\paragraph{Vide\'ok\'artya driver}
L\'asd \ref{gpudriver}.
Amennyiben a Vulkan SDK (\ref{vulkansdk}) telep\'it\'esre ker\"ul, abban megtal\'alhat\'o a bet\"olt\H o r\'eteg.
%A haszn\'alt Linux disztrib\'uci\'o csomagjai k\"oz\"ott \'erdemes a \code{vulkan} kulcssz\'ora r\'akeresni \'es a haszn\'alt hardvernek megfelel\"o csomagot/csomagokat feltelep\'iteni.

\paragraph{GLFW}
A megjelen\'it\H o ablak kezel\'es\'ere a GLFW k\"onyvt\'arat haszn\'alom. Ez szint\'en telep\'ithet\H o csomag a legt\"obb Linux disztrib\'uci\'oban, vagy a \href{http://www.glfw.org/download.html}{weboldalukr\'ol} let\"olthet\H o \'es telep\'it\'esi \'utmutat\'as is megtal\'alhat\'o. 

\paragraph{PortAudio}
A let\"olt\'ese t\"ort\'enhet csomagkeze\H ob\H ol, vagy a \href{http://www.portaudio.com/download.html}{weboldalukr\'ol}.

\paragraph{GLM}
Az OpenGL-b\H ol ismer\H os matematikai k\"onyvt\'ar. 
A telep\'it\'ese t\"ort\'enhet csomagkezel\H on kereszt\"ul, vagy mivel csak fejl\'eceket tartalmaz\'o k\"onyvt\'ar, a megfelel\H o helyre let\"olt\'essel.

\paragraph{Vulkan SDK}\label{vulkansdk}
\subparagraph{Bevezet\H o}
A LunarG Vulkan SDK sz\'amos elengedhetetlen eszk\"ozt tartalmaz. K\"ozt\"uk a sz\"uks\'eges fejl\'eceket, a standard valid\'aci\'os r\'etegeket, debuggol\'o eszk\"oz\"oket \'es a Vulkan f\"uggv\'enyek bet\"olt\H oj\'et.
\subparagraph{Telep\'it\'es}
A haszn\'alt Linux disztrib\'uci\'o csomagjai k\"oz\"ott val\'osz\'in\"uleg megtal\'alhat\'oak fejleszt\'eshez sz\"uks\'eges csomagok. 
\newline
Ellenkez\H o esetben a weboldalr\'ol (\url{https://vulkan.lunarg.com/sdk/home}) let\"olthet\H o a legfrissebb SDK verzi\'o.
A script futtat\'asa ut\'an lesz egy \code{VulkanSDK} mappa az adott k\"onyvt\'arban.
Az ebben tal\'alhat\'o \code{Getting\_Started.html} f\'ajl j\'o kiindul\'opont a tov\'abbiakhoz.
A \code{setup-env.sh} script automatikusan be\'all\'itja a sz\"uks\'eges k\"ornyezeti v\'altoz\'okat.
\subparagraph{Telep\'it\'es tesztel\'ese}
Ezek ut\'an a \code{vulkaninfo}\footnote{A \code{vulkaninfo --html} paranccsal egy k\"onnyeben b\"ong\'eszhet\H o weboldalt kapunk kimenetnek.} parancsot kiadva inform\'aci\'okat kaphatunk a rendszer\"unk k\'epess\'egeir\H ol, illetve tesztelhetj\"uk az SDK telep\'it\'es\'enek sikeress\'eg\'et. 
\'Erdemes lehet tov\'abb\'a p\'eldaprogramokat is lefuttatni. A \code{build\_examples.sh} script leford\'itja a p\'eldaprogramokat, amik ut\'ana a \code{examples/build/} k\"onyvt\'arban megtal\'alhat\'oak. (\code{cube, cubepp})

\subsubsection{Projekt strukt\'ura}
A gy\"ok\'er k\"onyvt\'ar tartalma:
\begin{itemize}
	\item \code{bin}: A leford\'itott futtathat\'o \'allom\'anyok
	\item \code{doc}: Ennek a dokumentumnak a \LaTeX forr\'asf\'ajljai
	\item \code{include}: Az alkalmaz\'as header f\'ajljai (\code{*.h})
	\item \code{src}: Az alkalmaz\'as forr\'asf\'ajljai
	\item \code{test}: Tesztel\'eshez haszn\'alt f\'ajlok
	\item \code{.clang\_complete}: Sz\"ovegszerkeszt\H oh\"oz autocomplete \'es linter plugin configur\'aci\'os f\'ajlja
	\item \code{.git}, \code{.gitignore}: Verzi\'okezel\H o f\'ajljai
	\item \code{README.md}: GitHub-ra "olvass el" f\'ajl
\end{itemize}

\paragraph{\code{bin} k\"onyvt\'ar}
Itt tov\'abbi k\'et alk\"onyvt\'ar tal\'alhat\'o: \code{debug} \'es \code{release}. Az el\"obbiben a debug m\'odban ford\'itott alkalmaz\'as, tal\'alhat\'o, amely annyit jelent, hogy gener\'al a \code{gdb} debuggernek plusz inform\'aci\'okat, illetve enged\'elyezi a Vulkan SDK valid\'aci\'os r\'etegeit.
Az ut\'obbiba a release m\'odban ford\'itott alkalmaz\'as ker\"ul. Ez valamivel gyorsabb fut\'ast jelent, cser\'eben az esetleges bugok eset\'en az alkalmaz\'as jobb esetben fut tov\'abb, rosszabb esetben \"osszeomlik, mindenf\'ele inform\'aci\'o a mi\'ertj\'er\H ol mell\H ozve.(\ref{compileoptions})

\paragraph{\code{doc} k\"onyvt\'ar}
A dokument\'aci\'o alapj\'aul az itt tal\'alhat\'o GitHub repository szolg\'alt: \url{https://github.com/shdnx/ELTE-LaTeX-Thesis-Base}

Amiben kieg\'esz\"ult az a projekt:
\begin{itemize}
	\item A \code{szakdolgozat.tex} f\'ajlba \'uj parancsokat vettem fel.
	\item Az \code{uml} k\"onyvt\'arban egy StarUML projekt, a \code{uml/png} k\"onyvt\'arban pedig az UML diagramok tal\'alhat\'oak \code{png} form\'atumban.
	\item A \code{tex} mapp\'aban a tartalmat tov\'abb daraboltam, hogy kisebb, k\"onnyebben \'atl\'athat\'o f\'ajlok rem\'eny\'eben.
		\begin{itemize}
			\item A \code{tex/dev} mapp\'aban tal\'alhat\'oak a fejleszt\H oi dokument\'aci\'o egyes r\'eszei.
		\end{itemize}
\end{itemize}

\paragraph{\code{src} k\"onyvt\'ar}
Ebben a k\"onyvt\'arban tal\'alhat\'oak a ford\'it\'asi egys\'egek.
Tov\'abba itt vannak m\'eg a shaderek is az \code{src/shaders} k\"onyvt\'arban.
A shaderekr\H ol tov\'abbiakat a \ref{shadercompilation} r\'eszben lehet olvasni.

\subsubsection{Ford\'it\'as}
A program gy\"ok\'erk\"onyvt\'ar\'aban a \code{make} parancs kiad\'as\'aval a k\"ovetkez\H ok t\"ort\'ennek:
\begin{enumerate}
	\item Leford\'itjuk a shadereket (ld. \ref{shadercompilation})
	\item Leford\'itjuk a programot (ld. \ref{compileoptions})
\end{enumerate}

\paragraph{Ford\'it\'o v\'alaszt\'asa}
A \code{Makefile} elej\'en a \code{COMPILER} v\'altoz\'ot a k\'iv\'ant ford\'it\'ora kell \'all\'itani (\code{g++/clang++}).
\subsubsection{Ford\'it\'asi param\'eterek}\label{compileoptions}
A teljes parancs (a \code{Makefile}-b\'ol): 
\begin{itemize}
	\item Debug verzi\'o:
		\$(COMPILER) -o \$(DTARGET)/\$(OUTPUT\_NAME) \$(SOURCES) \$(CFLAGS) \$(INCLUDE) \$(LDFLAGS) -DDEBUG -g -ggdb -Og
		\begin{itemize}
			\item \code{-o} ut\'an a kimeneti \'allom\'any van megadva
			\item \code{-DDEBUG}: Defini\'alja a \code{DEBUG} szimb\'olumot.
			\item \code{-g -ggdb}: Ford\'itson debuggol\'o szimb\'olumokat a futtathat\'o \'allom\'anyba.
			\item \code{-Og}: Olyan optimaliz\'aci\'okat v\'egez, ami nem akad\'alyozza a debuggol\'ast.
		\end{itemize}
	\item Release verzi\'o:
		\$(COMPILER) -o \$(TARGET)/\$(OUTPUT\_NAME) \$(SOURCES) \$(CFLAGS) \$(INCLUDE) \$(LDFLAGS) -DNDEBUG -O3
		\begin{itemize}
			\item \code{-o} ut\'an a kimeneti \'allom\'any van megadva
			\item \code{-DNDEBUG}: Defini\'alja az \code{NDEBUG} szimb\'olumot.
			\item \code{-O3}: 3-as szint\H u optimaliz\'aci\'o. \href{https://gcc.gnu.org/onlinedocs/gcc/Optimize-Options.html}{C++ optimaliz\'aci\'os be\'all\'it\'asok}
		\end{itemize}
\end{itemize}
Itt a k\"ulonb\"oz\H o v\'altoz\'ok:
\begin{itemize}
	\item \code{COMPILER}: A v\'alasztott ford\'it\'o (\code{g++} vagy \code{clang++})
	\item \code{TARGET/DTARGET}: A c\'elmappa (\code{bin/debug} vagy \code{bin/release}) 
	\item \code{OUTPUT\_NAME}: A futtathat\'o \'allom\'any neve.
	\item \code{SOURCES}: Az \code{src} k\"onyvt\'ar \code{.cpp} f\'ajljai.
	\item \code{CFLAGS}: Milyen ford\'it\'asi param\'eterek legyenek megadva.
		A program C++11-es szabv\'any szerint k\'esz\"ul (\code{-std=c++11}) \'es szeretn\'ek tudni az \"osszes figyelmeztet\'esr\H ol, amit a ford\'it\'o ta\'al. (\code{-Wall})
	\item \code{INCLUDE}: Hol tal\'alja a ford\'it\'o a header f\'ajlokat. 
	\item \code{LDFLAGS}: Hol tal\'alja a linker a felhaszn\'alt k\"onyvt\'arak met\'odusainak defin\'ici\'oit. 
\end{itemize}

\subsubsection{Shaderek}\label{shadercompilation}
A Vulkan API \href{https://www.khronos.org/spir/}{SPIR-V} shadereket haszn\'al. Mivel ez egy b\'ajtk\'od form\'atum, \'igy a shaderek manu\'alis \'ir\'asa \'es olvas\'asa szokatlan lehet. 
Szerencs\'ere a Khronos Group lefejlesztett egy ford\'it\'ot, ami az OpenGL-b\H ol ismer\H os GLSL shadert SPIR-V shaderr\'e alak\'itja. A LunarG SDK-ban megtal\'alhat\'o ez a program, a 
\code{glslangValidator}. \newline
A GLSL shaderek ford\'it\'asa \'igy t\"obbf\'elek\'eppen is t\"ort\'enhet:
\begin{itemize}
	\item \code{make shaders} \newline
		A \code{Makefile}-ban defini\'alt parancs, leford\'tja az \"osszes shadert a \code{src/shaders} mapp\'aban.
	\item \code{glslangValidator -V <shader el\'er\'esi \'utja>.<shader t\'ipusa>} \newline
		A \code{-V} kapcsol\'o mondja meg, hogy gener\'aljon is SPIR-V shadert, an\'elk\"ul csak ellen\H orzi a k\'odot, hogy megfelel-e az implement\'alt szabv\'anynak.
		A shader t\'ipusa alapj\'an az eredm\'enyezett f\'ajl \code{<shader t\'ipusa>.spv} alak\'u. A programba ez alapj\'an a n\'ev lesz bet\"oltve.
		\textbf{Fontos!} A SPIR-V shaderek az \'eppen aktu\'alis k\"onyvt\'arba ker\"ulnek. 
\end{itemize}



\subsection{Tesztel\'es}
A program vizu\'alis volta miatt a tesztel\'es a sz\'am\'it\'asi r\'eszre szor\'itkozott, hogy a diszkr\'et Fourier-transzform\'aci\'o helyes eredm\'enyt sz\'amol-e.

Az \code{AudioHandler} oszt\'alyban van lehet\H os\'eg \code{WAV} f\'ajlt bet\"olteni, illetve megadott frekvenci\'aj\'u hangot gener\'alni.

Gener\'alt hang eset\'en az eredm\'enyk\'ent kapott frekvencia megegyezik a megadott frekvenci\'aval.

Hang f\'ajl bet\"olt\'ese eset\'en a vide\'ok\'arty\'an v\'egzett DFT eredm\'enye megegyezik\footnote{Mivel lebeg\H opontos sz\'amokr\'ol van sz\'o, itt a megegyezik azt jelent\'i, hogy egy $\varepsilon>0$ sugar\'u k\"ornyezet\'en bel\"ul van. Ez a gyakorlatban olyan $10^{-5}$ nagys\'agrendet jelentett.} a processzoron futtatott DFT algoritmus eredm\'eny\'evel.

TODO: tesztf\'ajlok fut\'as\'ar\'ol k\'ep.

Mivel az alkalmaz\'asn\'al fontos szerepet j\'atszik a kis k\'es\'es\H u megjelen\'it\'es, a k\"ovetkez\H o r\'eszben m\'er\'esek eredm\'enyeit r\'eszletezem.


\include{tex/summary}

\begin{thebibliography}{9}
	\bibitem{citekey}content...
\end{thebibliography}


\end{document}
