
\section{Bevezet\H o}
\paragraph{} 
Inform\'aci\'o minden\"utt jelen van k\"or\"ul\"ott\"unk. A technol\'ogia fejl\H od\'es\'evel egyre t\"obbet \'es t\"obbet tudtunk bel\H ole digitaliz\'alni. Ez\'altal az adat rendelkez\'esre \'all, a k\'erd\'es az marad, hogy ezt hogyan tudjuk felhaszn\'alni.

Az ut\'obbi id\H oben m\'egink\'abb el\H ot\'erbe ker\"ult a vide\'ok\'arty\'ak nem csak j\'at\'ekokban val\'o kihaszn\'al\'asa, hanem nagym\'ert\'ekben t\"obbsz\'alas\'ithat\'o probl\'em\'ak nagy bemenetre val\'o felhaszn\'al\'asa. 
Ugyanakkor a nagyobb sz\'am\'it\'asi kapacit\'as nem sokat \'er, ha a programoz\'o nem tudja kihaszn\'alni. Az OpenGL egyre ink\'abb alulmaradt a DirectX-szel szemben, mind haszn\'alhat\'os\'ag, mint teljes\'itm\'eny szempontj\'ab\'ol.

Az OpenGL-\'ert felel\H os \href{https://www.khronos.org/}{Khronos Group} a 2015-\"os GDC\footnote{\href{http://www.gdconf.com/}{Game Developers Conference}}-n bejelentette a Vulkan-t, mint egy alacsonyszint\H u, cross-platform sz\'am\'it\'asi \'es 3D grafika API.
%\paragraph{Motiv\'aci\'o}
