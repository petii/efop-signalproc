
\subsection{A feladat r\'eszletes le\'ir\'asa}
\subsubsection{Modell}
Adott egy hangull\'am, ami a l\'egnyom\'as v\'altoz\'asa az id\H o f\"uggv\'eny\'eben:
\(s: \mathbb{T} \rightarrow \mathbb{P} \) \newline
A modell\"unkben legyen $\mathbb{T}:=[0,+\infty), \mathbb{P}:=[-1,1] $ \newline
A hanghull\'am, mint \"osszetett rezg\'es fel\'irhat\'o k\"ul\"onb\"oz\H o amplit\'ud\'oj\'u, f\'azis\'u \'es frekvenci\'aj\'u harmonikus rezg\'esek (r\'eszhangok) line\'aris kombin\'aci\'ojak\'ent, azaz: \newline
\( s(t) = \sum_i r_i\cdot\sin{(f_i\cdot t + \varphi_i)} \), ahol $r_i$ az amplit\'ud\'oja, $f_i$ a frekvenci\'aja, $\varphi_i$ pedig a f\'azisa az $i$-edik r\'eszhangnak.
\subsubsection{Specifik\'aci\'o}
\paragraph{Bemenet}
A modellben szerepl\H o $s$ f\"uggv\'enyb\H ol $\Delta t$ id\H o alatt egy p\'aros $N$ elem\H u mint\'at v\'etelez\"unk: $s_0,\dots,s_{N-1}$, \'es \newline
\( \begin{array}{rcl}
s_i := s(x_i) & \text{, ahol } & i\in[0..N-1],\\
&&x_i\in\{ t_0 = x_0 < x_1 < \dots < x_{N-1} = t_0 + \Delta t \} \text{, tov\'abb\'a } \\
&&x_j - x_{j-1} = \frac{\Delta t}{N}\quad (j\in[1..N-1] ) 
\end{array} \)
\paragraph{Kimenet}
Legyen $\mathbb{F}:=\left[ 0..\frac{N}{2}-1 \right]$ \newline 
Ekkor a kimenet egy olyan $R_{s,t_0}:\mathbb{F} \rightarrow [0,+\infty)$ f\"uggv\'eny, amely a frekvencia f\"uggv\'eny\'eben megadja az $s\rvert_{[t_0,t_0+\Delta t]}$ hanghull\'amban az adott frekvenci\'aj\'u r\'eszhang amplit\'ud\'oj\'at, azaz $R_{s,t_0}(f_i) = r_i\quad (i\in [0..N-1])$.

\subparagraph{A r\'eszhang amplit\'ud\'oj\'anak sz\'amol\'asa}
Az $s_0,\dots,s_{N-1}$ mint\'an diszkr\'et Fourier-transzform\'aci\'ot v\'egezve eredm\'enyk\'ent $N$ darab komplex sz\'amot kapunk. Legyen az eredm\'eny $ \vecc{h'}=(h_0,\dots,h_{N-1})\in\mathbb{C^N}$. \newline
Maga a transzform\'aci\'o a k\"ovetkez\H ok\'eppen van defini\'alva:
\[
h_i = \sum_{j=0}^{N-1} s_j \cdot \left[ \cos\left( \frac{2\pi i j}{N} \right) - \imath\cdot\sin\left( \frac{2\pi i j}{N} \right) \right] \text{, ahol $\imath$ az imagin\'arius egys\'eg}
\]
A $h_i$ sz\'amb\'ol az $i$-edik r\'eszhang al\'abbi tulajdons\'agait tudhatjuk meg:
\begin{itemize}
	\item $r_i$ amplit\'ud\'o: $|h_i|=\sqrt{\Re{(h_i)}^2 + \Im{(h_i)}^2}$, a komplex sz\'am abszol\'ut \'ert\'eke vagy nagys\'aga
	\item $f_i$ frekvencia: $i\ \hz$ 
	\item $\varphi_i$ f\'azis: $\arg (h_i)$, a komplex sz\'am ir\'anysz\"oge vagy arkusza
\end{itemize}

Val\'os minta eset\'en az eredm\'enyre az al\'abbi szimmetria teljes\"ul:
\[
h_i = \conj{h_{N-1-i}} \quad (i\in\left[0..\infrac{(N)}{2}-1\right])\text{, ahol $\conj{x}\ (x\in\mathbb C)$ a komplex konjug\'altat jel\"oli}
\]
A konjug\'al\'as az amplit\'ud\'on nem v\'altoztat, \'igy az ism\'etl\H od\'es\'et elker\'ulve az eredm\'enyt a $\vecc{h}=(h_0,\dots,h_{\frac{N}{2}-1})\in\mathbb{C}^{\frac{N}{2}}$ vektor tartalmazza.

Ezek alapj\'an az $R_{s,t}$ f\"uggv\'eny a k\"ovetkez\H ok\'epp adhat\'o meg:
\[ \begin{aligned}
R_{s,t}(f_i):=\left| \scalmul{\vecc{h}}{\vecc{e}_i} \right|
\text{, ahol}\  & \scalmul{.}{.}\text{ a skal\'aris szorzat} \\
& \vecc{e}_i\text{ pedig az $i$-edik egys\'egvektor}
\end{aligned}
\]
azaz
\[ R_{s,t}(f_i) = |h_i| = \sqrt{\Re{(h_i)}^2 + \Im{(h_i)}^2}
\quad (i\in [0..\infrac{N}{2}-1]) \]


\paragraph{\'Abr\'azol\'as}
Legyen $\mathbb{T}_{\Delta t}:=\left\{ k\cdot\Delta t : k\in\mathbb{N} \right\} \subset \mathbb{T}$ az id\H o $\Delta t$ szerinti diszkretiz\'al\'asa. \newline
Mag\'ahoz az $s$ hanghull\'amhoz tartoz\'o spektrogram az al\'abbi f\"uggv\'ennyel \'irhat\'o le:
\[ \begin{aligned}[rl]
S_s&:\mathbb{T}_{\Delta t}\times\mathbb{F} \rightarrow [0,+\infty), \\
S_s&(t,f):=R_{s,t}(f) \quad \left( (t,f)\in \mathbb{T}_{\Delta t}\times\mathbb{F} \right)
\end{aligned}
\]
Ekkor az \'abr\'azol\'as l\'enyeg\'eben megfelel az $S_s$ f\"uggv\'eny grafikonj\'anak, azaz
\[
\graf S_s = \left\{ \big( (t,f) ,S_s(t,f) \big) : (t,f)\in \mathbb{T}_{\Delta t}\times\mathbb{F} \right\}
\]

