
\subsection{Fejleszt\'es}
\subsubsection{El\H ok\"ovetelm\'enyek}
\paragraph{Vide\'ok\'artya driver}
L\'asd \ref{gpudriver}.
Amennyiben a Vulkan SDK (\ref{vulkansdk}) telep\'it\'esre ker\"ul, abban megtal\'alhat\'o a bet\"olt\H o r\'eteg.
%A haszn\'alt Linux disztrib\'uci\'o csomagjai k\"oz\"ott \'erdemes a \code{vulkan} kulcssz\'ora r\'akeresni \'es a haszn\'alt hardvernek megfelel\"o csomagot/csomagokat feltelep\'iteni.

\paragraph{GLFW}
A megjelen\'it\H o ablak kezel\'es\'ere a GLFW k\"onyvt\'arat haszn\'alom. Ez szint\'en telep\'ithet\H o csomag a legt\"obb Linux disztrib\'uci\'oban, vagy a \href{http://www.glfw.org/download.html}{weboldalukr\'ol} let\"olthet\H o \'es telep\'it\'esi \'utmutat\'as is megtal\'alhat\'o. 

\paragraph{PortAudio}
A let\"olt\'ese t\"ort\'enhet csomagkeze\H ob\H ol, vagy a \href{http://www.portaudio.com/download.html}{weboldalukr\'ol}.

\paragraph{GLM}
Az OpenGL-b\H ol ismer\H os matematikai k\"onyvt\'ar. 
A telep\'it\'ese t\"ort\'enhet csomagkezel\H on kereszt\"ul, vagy mivel csak fejl\'eceket tartalmaz\'o k\"onyvt\'ar, a megfelel\H o helyre let\"olt\'essel.

\paragraph{Vulkan SDK}\label{vulkansdk}
\subparagraph{Bevezet\H o}
A LunarG Vulkan SDK sz\'amos elengedhetetlen eszk\"ozt tartalmaz. K\"ozt\"uk a sz\"uks\'eges fejl\'eceket, a standard valid\'aci\'os r\'etegeket, debuggol\'o eszk\"oz\"oket \'es a Vulkan f\"uggv\'enyek bet\"olt\H oj\'et.
\subparagraph{Telep\'it\'es}
A haszn\'alt Linux disztrib\'uci\'o csomagjai k\"oz\"ott val\'osz\'in\"uleg megtal\'alhat\'oak fejleszt\'eshez sz\"uks\'eges csomagok. 
\newline
Ellenkez\H o esetben a weboldalr\'ol (\url{https://vulkan.lunarg.com/sdk/home}) let\"olthet\H o a legfrissebb SDK verzi\'o.
A script futtat\'asa ut\'an lesz egy \code{VulkanSDK} mappa az adott k\"onyvt\'arban.
Az ebben tal\'alhat\'o \code{Getting\_Started.html} f\'ajl j\'o kiindul\'opont a tov\'abbiakhoz.
A \code{setup-env.sh} script automatikusan be\'all\'itja a sz\"uks\'eges k\"ornyezeti v\'altoz\'okat.
\subparagraph{Telep\'it\'es tesztel\'ese}
Ezek ut\'an a \code{vulkaninfo}\footnote{A \code{vulkaninfo --html} paranccsal egy k\"onnyeben b\"ong\'eszhet\H o weboldalt kapunk kimenetnek.} parancsot kiadva inform\'aci\'okat kaphatunk a rendszer\"unk k\'epess\'egeir\H ol, illetve tesztelhetj\"uk az SDK telep\'it\'es\'enek sikeress\'eg\'et. 
\'Erdemes lehet tov\'abb\'a p\'eldaprogramokat is lefuttatni. A \code{build\_examples.sh} script leford\'itja a p\'eldaprogramokat, amik ut\'ana a \code{examples/build/} k\"onyvt\'arban megtal\'alhat\'oak. (\code{cube, cubepp})

\subsubsection{Projekt strukt\'ura}
A gy\"ok\'er k\"onyvt\'ar tartalma:
\begin{itemize}
	\item \code{bin}: A leford\'itott futtathat\'o \'allom\'anyok
	\item \code{doc}: Ennek a dokumentumnak a \LaTeX forr\'asf\'ajljai
	\item \code{include}: Az alkalmaz\'as header f\'ajljai (\code{*.h})
	\item \code{src}: Az alkalmaz\'as forr\'asf\'ajljai
	\item \code{test}: Tesztel\'eshez haszn\'alt f\'ajlok
	\item \code{.clang\_complete}: Sz\"ovegszerkeszt\H oh\"oz autocomplete \'es linter plugin configur\'aci\'os f\'ajlja
	\item \code{.git}, \code{.gitignore}: Verzi\'okezel\H o f\'ajljai
	\item \code{README.md}: GitHub-ra "olvass el" f\'ajl
\end{itemize}

\paragraph{\code{bin} k\"onyvt\'ar}
Itt tov\'abbi k\'et alk\"onyvt\'ar tal\'alhat\'o: \code{debug} \'es \code{release}. Az el\"obbiben a debug m\'odban ford\'itott alkalmaz\'as, tal\'alhat\'o, amely annyit jelent, hogy gener\'al a \code{gdb} debuggernek plusz inform\'aci\'okat, illetve enged\'elyezi a Vulkan SDK valid\'aci\'os r\'etegeit.
Az ut\'obbiba a release m\'odban ford\'itott alkalmaz\'as ker\"ul. Ez valamivel gyorsabb fut\'ast jelent, cser\'eben az esetleges bugok eset\'en az alkalmaz\'as jobb esetben fut tov\'abb, rosszabb esetben \"osszeomlik, mindenf\'ele inform\'aci\'o a mi\'ertj\'er\H ol mell\H ozve.(\ref{compileoptions})

\paragraph{\code{doc} k\"onyvt\'ar}
A dokument\'aci\'o alapj\'aul az itt tal\'alhat\'o GitHub repository szolg\'alt: \url{https://github.com/shdnx/ELTE-LaTeX-Thesis-Base}

Amiben kieg\'esz\"ult az a projekt:
\begin{itemize}
	\item A \code{szakdolgozat.tex} f\'ajlba \'uj parancsokat vettem fel.
	\item Az \code{uml} k\"onyvt\'arban egy StarUML projekt, a \code{uml/png} k\"onyvt\'arban pedig az UML diagramok tal\'alhat\'oak \code{png} form\'atumban.
	\item A \code{tex} mapp\'aban a tartalmat tov\'abb daraboltam, hogy kisebb, k\"onnyebben \'atl\'athat\'o f\'ajlok rem\'eny\'eben.
		\begin{itemize}
			\item A \code{tex/dev} mapp\'aban tal\'alhat\'oak a fejleszt\H oi dokument\'aci\'o egyes r\'eszei.
		\end{itemize}
\end{itemize}

\paragraph{\code{src} k\"onyvt\'ar}
Ebben a k\"onyvt\'arban tal\'alhat\'oak a ford\'it\'asi egys\'egek.
Tov\'abba itt vannak m\'eg a shaderek is az \code{src/shaders} k\"onyvt\'arban.
A shaderekr\H ol tov\'abbiakat a \ref{shadercompilation} r\'eszben lehet olvasni.

\subsubsection{Ford\'it\'as}
A program gy\"ok\'erk\"onyvt\'ar\'aban a \code{make} parancs kiad\'as\'aval a k\"ovetkez\H ok t\"ort\'ennek:
\begin{enumerate}
	\item Leford\'itjuk a shadereket (ld. \ref{shadercompilation})
	\item Leford\'itjuk a programot (ld. \ref{compileoptions})
\end{enumerate}

\paragraph{Ford\'it\'o v\'alaszt\'asa}
A \code{Makefile} elej\'en a \code{COMPILER} v\'altoz\'ot a k\'iv\'ant ford\'it\'ora kell \'all\'itani (\code{g++/clang++}).
\subsubsection{Ford\'it\'asi param\'eterek}\label{compileoptions}
A teljes parancs (a \code{Makefile}-b\'ol): 
\begin{itemize}
	\item Debug verzi\'o:
		\$(COMPILER) -o \$(DTARGET)/\$(OUTPUT\_NAME) \$(SOURCES) \$(CFLAGS) \$(INCLUDE) \$(LDFLAGS) -DDEBUG -g -ggdb -Og
		\begin{itemize}
			\item \code{-o} ut\'an a kimeneti \'allom\'any van megadva
			\item \code{-DDEBUG}: Defini\'alja a \code{DEBUG} szimb\'olumot.
			\item \code{-g -ggdb}: Ford\'itson debuggol\'o szimb\'olumokat a futtathat\'o \'allom\'anyba.
			\item \code{-Og}: Olyan optimaliz\'aci\'okat v\'egez, ami nem akad\'alyozza a debuggol\'ast.
		\end{itemize}
	\item Release verzi\'o:
		\$(COMPILER) -o \$(TARGET)/\$(OUTPUT\_NAME) \$(SOURCES) \$(CFLAGS) \$(INCLUDE) \$(LDFLAGS) -DNDEBUG -O3
		\begin{itemize}
			\item \code{-o} ut\'an a kimeneti \'allom\'any van megadva
			\item \code{-DNDEBUG}: Defini\'alja az \code{NDEBUG} szimb\'olumot.
			\item \code{-O3}: 3-as szint\H u optimaliz\'aci\'o. \href{https://gcc.gnu.org/onlinedocs/gcc/Optimize-Options.html}{C++ optimaliz\'aci\'os be\'all\'it\'asok}
		\end{itemize}
\end{itemize}
Itt a k\"ulonb\"oz\H o v\'altoz\'ok:
\begin{itemize}
	\item \code{COMPILER}: A v\'alasztott ford\'it\'o (\code{g++} vagy \code{clang++})
	\item \code{TARGET/DTARGET}: A c\'elmappa (\code{bin/debug} vagy \code{bin/release}) 
	\item \code{OUTPUT\_NAME}: A futtathat\'o \'allom\'any neve.
	\item \code{SOURCES}: Az \code{src} k\"onyvt\'ar \code{.cpp} f\'ajljai.
	\item \code{CFLAGS}: Milyen ford\'it\'asi param\'eterek legyenek megadva.
		A program C++11-es szabv\'any szerint k\'esz\"ul (\code{-std=c++11}) \'es szeretn\'ek tudni az \"osszes figyelmeztet\'esr\H ol, amit a ford\'it\'o ta\'al. (\code{-Wall})
	\item \code{INCLUDE}: Hol tal\'alja a ford\'it\'o a header f\'ajlokat. 
	\item \code{LDFLAGS}: Hol tal\'alja a linker a felhaszn\'alt k\"onyvt\'arak met\'odusainak defin\'ici\'oit. 
\end{itemize}

\subsubsection{Shaderek}\label{shadercompilation}
A Vulkan API \href{https://www.khronos.org/spir/}{SPIR-V} shadereket haszn\'al. Mivel ez egy b\'ajtk\'od form\'atum, \'igy a shaderek manu\'alis \'ir\'asa \'es olvas\'asa szokatlan lehet. 
Szerencs\'ere a Khronos Group lefejlesztett egy ford\'it\'ot, ami az OpenGL-b\H ol ismer\H os GLSL shadert SPIR-V shaderr\'e alak\'itja. A LunarG SDK-ban megtal\'alhat\'o ez a program, a 
\code{glslangValidator}. \newline
A GLSL shaderek ford\'it\'asa \'igy t\"obbf\'elek\'eppen is t\"ort\'enhet:
\begin{itemize}
	\item \code{make shaders} \newline
		A \code{Makefile}-ban defini\'alt parancs, leford\'tja az \"osszes shadert a \code{src/shaders} mapp\'aban.
	\item \code{glslangValidator -V <shader el\'er\'esi \'utja>.<shader t\'ipusa>} \newline
		A \code{-V} kapcsol\'o mondja meg, hogy gener\'aljon is SPIR-V shadert, an\'elk\"ul csak ellen\H orzi a k\'odot, hogy megfelel-e az implement\'alt szabv\'anynak.
		A shader t\'ipusa alapj\'an az eredm\'enyezett f\'ajl \code{<shader t\'ipusa>.spv} alak\'u. A programba ez alapj\'an a n\'ev lesz bet\"oltve.
		\textbf{Fontos!} A SPIR-V shaderek az \'eppen aktu\'alis k\"onyvt\'arba ker\"ulnek. 
\end{itemize}
