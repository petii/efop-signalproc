
\subsection{Fejleszt\'es}
\subsubsection{El\H ok\"ovetelm\'enyek}
\paragraph{Vide\'ok\'artya driver}
L\'asd \ref{gpudriver}.
%A haszn\'alt Linux disztrib\'uci\'o csomagjai k\"oz\"ott \'erdemes a \code{vulkan} kulcssz\'ora r\'akeresni \'es a haszn\'alt hardvernek megfelel\"o csomagot/csomagokat feltelep\'iteni.

\paragraph{GLFW}
A megjelen\'it\H o ablak kezel\'es\'ere a GLFW k\"onyvt\'arat haszn\'alom. Ez szint\'en telep\'ithet\H o csomag a legt\"obb Linux disztrib\'uci\'oban, vagy a weboldalukr\'ol let\"olthet\H o \'es telep\'it\'esi \'utmutat\'as is megtal\'alhat\'o. \url{http://www.glfw.org/download.html}

\paragraph{PortAudio}
Csomagkezel\H o

\paragraph{GLM}
Az OpenGL-b\H ol ismer\H os matematikai k\"onyvt\'ar. 
A telep\'it\'ese t\"ort\'enhet csomagkezel\H on kereszt\"ul, vagy mivel csak fejl\'eceket tartalmaz\'o k\"onyvt\'ar, a megfelel\H o helyre let\"olt\'essel.

\paragraph{Vulkan SDK}
\subparagraph{Bevezet\H o}
A LunarG Vulkan SDK sz\'amos elengedhetetlen eszk\"ozt tartalmaz. K\"ozt"uk a sz\"uks\'eges fejl\'eceket, a standard valid\'aci\'os r\'etegeket, debuggol\'o eszk\"oz\"oket \'es a Vulkan f\"uggv\'enyek bet\"olt\H oj\'et. 
\subparagraph{Telep\'it\'es}
A haszn\'alt Linux disztrib\'uci\'o csomagjai k\"oz\"ott val\'osz\'in\"uleg megtal\'alhat\'oak fejleszt\'eshez sz\"uks\'eges csomagok. 
\newline
Ellenkez\H o esetben a weboldalr\'ol (\url{https://vulkan.lunarg.com/sdk/home}) let\"olthet\H o a legfrissebb SDK verzi\'o.
A script futtat\'asa ut\'an lesz egy \code{VulkanSDK} mappa az adott k\"onyvt\'arban.
Az ebben tal\'alhat\'o \code{Getting\_Started.html} f\'ajl j\'o kiindul\'opont a tov\'abbiakhoz.
A \code{setup-env.sh} script automatikusan be\'all\'itja a sz\"uks\'eges k\"ornyezeti v\'altoz\'okat.
\subparagraph{Tesztel\'es}
Ezek ut\'an a \code{vulkaninfo}\footnote{A \code{vulkaninfo --html} paranccsal egy k\"onnyeben b\"ong\'eszhet\H o weboldalt kapunk kimenetnek.} parancsot kiadva inform\'aci\'okat kaphatunk a rendszer\"unk k\'epess\'egeir\H ol, illetve tesztelhetj\"uk az SDK telep\'it\'es\'enek sikeress\'eg\'et. 
\'Erdemes lehet tov\'abb\'a p\'eldaprogramokat is lefuttatni. A \code{build\_examples.sh} script leford\'itja a p\'eldaprogramokat, amik ut\'ana a \code{examples/build/} k\"onyvt\'arban megtal\'alhat\'oak. (\code{cube, cubepp})

\subsubsection{Projekt szerkezet}
Mapp\'ak

\subsubsection{Ford\'it\'as}
A program gy\"ok\'erk\"onyvt\'ar\'aban vagyunk
a \code{make} parancs kiad\'as\'aval a k\"ovetkez\H ok t\"ort\'ennek:
\begin{enumerate}
	\item Leford\'itjuk a shadereket (ld. \ref{shadercompilation})
	\item Leford\'itjuk a programot (ld. \ref{compileoptions})
\end{enumerate}

\paragraph{Ford\'it\'o v\'alaszt\'asa}
A \code{Makefile} elej\'en a \code{COMPILER} v\'altoz\'ot a k\'iv\'ant ford\'it\'ora kell \'all\'itani (\code{g++/clang++}).
\subsubsection{Ford\'it\'asi param\'eterek}\label{compileoptions}
A teljes parancs (a \code{Makefile}-b\'ol): 
\begin{itemize}
	\item Debug verzi\'o
		%\code{\$(COMPILER) -o \$(DTARGET)/\$(OUTPUT_NAME) \$(SOURCES) \$(CFLAGS) \$(INCLUDE) \$(LDFLAGS) -DDEBUG -g -ggdb}
	\item Release verzi\'o
		%\code{\$(COMPILER) -o \$(TARGET)/\$(OUTPUT_NAME) \$(SOURCES) \$(CFLAGS) \$(INCLUDE) \$(LDFLAGS) -DNDEBUG }
\end{itemize}
Itt a k\"ulonb\"oz\H o v\'altoz\'ok:
\begin{itemize}
	\item \code{COMPILER}: A v\'alasztott ford\'it\'o (\code{gcc} vagy \code{clang})
	\item \code{TARGET/DTARGET}: A c\'elmappa (\code{bin/debug} vagy \code{bin/release}) 
	\item \code{OUTPUT\_NAME}: A futtathat\'o \'allom\'any neve
	\item \code{SOURCES}: Az \code{src} k\"onyvt\
	\item \code{CFLAGS}: 
	\item \code{INCLUDE}: 
	\item \code{LDFLAGS}: 
\end{itemize}

\subsubsection{Shader}\label{shadercompilation}
A Vulkan API \href{https://www.khronos.org/spir/}{SPIR-V} shadereket haszn\'al. Mivel ez egy b\'ajtk\'od form\'atum, \'igy a shaderek manu\'alis \'ir\'asa \'es olvas\'asa szokatlan lehet. 
Szerencs\'ere a Khronos Group lefejlesztett egy ford\'it\'ot, ami az OpenGL-b\H ol ismer\H os GLSL shadert SPIR-V shaderr\'e alak\'itja. A LunarG SDK-ban megtal\'alhat\'o ez a program, a 
\code{glslangValidator}. \newline
A GLSL shaderek ford\'it\'asa \'igy t\"obbf\'elek\'eppen is t\"ort\'enhet:
\begin{itemize}
	\item \code{make shaders} \newline
		A \code{Makefile}-ban defini\'alt parancs, leford\'tja az \"osszes shadert a \code{src/shaders} mapp\'aban.
	\item \code{glslangValidator -V src/shaders/<shader neve>.<shader t\'ipusa>} \newline
		A \code{-V} kapcsol\'o mondja meg, hogy gener\'aljon is SPIR-V shadert, an\'elk\"ul csak ellen\H orzi a k\'odot, hogy megfelel-e az implement\'alt szabv\'anynak.
		A shader t\'ipusa alapj\'an 
\end{itemize}
