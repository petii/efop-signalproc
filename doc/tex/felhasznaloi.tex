
\section{Felhaszn\'al\'oi dokument\'aci\'o}
\subsection{Feladat}
A program azt feladatot hivatott ell\'atni, hogy valamilyen audio bemenetr\"ol \'erkez\H o jelet feldolgozzon, majd ezt az eredm\'enyt vizu\'alisan megjelen\'itse. Mindezt \'ugy, hogy az id\H o ami eltelik a jel kiad\'asa \'es az eredm\'eny megjelen\'ese k\"oz\"ott minim\'alis legyen.
A megjelen\'it\'es egy h\'aromdimenzi\'os spektogramban nyilv\'anul meg. Az $x$-tengelyen az id\H o, az $y$-tengelyen a frekvencia, a $z$-tengelyen pedig az adott frekvenci\'aj\'u hull\'am amplit\'ud\'oja jelenik meg.

\subsection{Technol\'ogia}

\subsection{Hangfeldolgoz\'as}
A jel feldolgoz\'asa vide\'ok\'arty\'an sz\'amolt diszkr\'et Fourier-transzform\'aci\'oval t\"ort\'enik. A transzform\'aci\'o a hangjelet az id\H o f\"uggv\'eny\'eb\H ol a frekvencia f\"uggv\'eny\'eve alak\'itja. 

\subsection{Futtat\'as}

\subsection{Rendszerk\"ovetelm\'enyek}
A fejleszt\'es Linuxon, 64 bites AntergOS disztrib\'uci\'on zajlott. Mivel a felhaszn\'alt eszk\"oz\"ok mindegyike cross-platform\footnote{PortAudio, GLFW: Linux, Windows; Vulkan: Linux, Windows, Android}, \'igy elm\'eletben Windowson is ford\'ithat\'o \'es futtathat\'o a program\footnote{<Vulkan t\'amogatotts\'ag>}.
